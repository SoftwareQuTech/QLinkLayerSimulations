\documentclass{article}
\usepackage[utf8]{inputenc}

\title{Simulation Notes}
\author{Matthew Skrzypczyk}
\date{July 2018}

\begin{document}

\maketitle

\section{Simulation Options}
This document outlines various simulation parameters that can be used to test the EGP under various scenario regimes
\begin{itemize}
    \item $--network-config$: Path the the network configuration file to use for the simulation.
    \item $--results-path$: Directory to store the results under.  This will automatically create a timestamped subdirectory containing the simulation data.
    \item $--origin-bias$: Probability that the request comes from node A rather than node B.
    \item $--create-probability$: Probability that a CREATE request is submitted at a particular timestep
    \item $--min-pairs$: The minimum number of pairs to include in a CREATE request
    \item $ --max-pairs$: The maximum number of pairs that can be requested
    \item $--tmax-pair$: Maximum amount of time per pair (in seconds) in a request.  The total max\_time of a request is then $num\_pairs \times tmax\_pair$
    \item $--request-overlap$: Allow request submissions to overlap, this causes requests to be submitted before previous request's max\_time has expired.
    \item $--request-frequency$: Minimum amount of time (in seconds) between calls to CREATE
    \item $--num-requests$: Total number of requests to simulate
    \item $--enable-pdb$: Turn on PDB pre and post simulation
\end{itemize}

\section{Examples}
Here I will describe various simulation scenarios and appropriate command line arguments to the simulation script to simulate them.
\subsection{Example 1.}
A simulation where:
\begin{itemize}
    \item Requests have equal probability of being created at either node
    \item Requests are created with probability 0.5
    \item Requests strictly contain 2 pairs
    \item Total number of attempted requests is 100
    \item Only one request is active at any time
    \item Config file "network\_config.json" is used
    \item Store results in "results" directory
\end{itemize}

\begin{verbatim}
python3 simulation.py --network-config network_config.json --results-path
results --origin-bias 0.5 --create-probability 0.5 --min-pairs 2 --max-pairs 2
--num-requests 100
\end{verbatim}

\subsection{Example 2}
A simulation where:
\begin{itemize}
    \item Requests originate from node A with probability 0.2
    \item Requests are always created when attempted
    \item Requests contain pairs in the range [2,3]
    \item Requests are allowed to overlap
    \item Total number of attempted requests is 1
    \item Requests occur in a frequency of 0.05 seconds
    \item Config file "/home/user/network\_config.json" is used
    \item Store results in "results" directory
\end{itemize}

\begin{verbatim}
python3 simulation.py --network-config /home/user/network_config.json
--results-path results --origin-bias 0.2 --create-probability 1 --min-pairs 2
--max-pairs 3 --request-overlap --request-frequency 0.05 --num-requests 1
\end{verbatim}

\subsection{Example 3}
A simulation where:
\begin{itemize}
    \item Requests originate strictly from node B
    \item Requests are created with probability 0.8
    \item Requests contain only 1 pair
    \item Maximum time for a pair per request is 10 seconds
    \item Total number of attempted requests of 1000
    \item Requests are allowed to overlap
    \item Requests occur in a frequency of 2 seconds
    \item Config file "network\_config.json" is used
    \item Store results in "/home/user/simulation\_results" directory
\end{itemize}

\begin{verbatim}
python3 simulation.py --network-config network_config.json --results-path
/home/user/simulation_results --origin-bias 1 --create-probability 0.8
--min-pairs 1 --max-pairs 1 --request-overlap --request-frequency 2
--num-requests 1000 --tmax-pair 10
\end{verbatim}

\subsection{Example 4}
A simulation where:
\begin{itemize}
    \item Requests originate equally from both nodes
    \item Requests are created with probability 1
    \item Requests contain pairs in the range [10, 20]
    \item Maximum time for a pair per request is 100 seconds
    \item Total number of attempted requests of 1000
    \item Requests are allowed to overlap
    \item Requests occur in a frequency of 10 seconds
    \item Config file "network\_config.json" is used
    \item Store results in "results" directory
    \item Enable the debugger for inspection of local variables before and after simulation
\end{itemize}

\begin{verbatim}
python3 simulation.py --network-config network_config.json --results-path
results --origin-bias 0.5 --create-probability 1 --min-pairs 10
--max-pairs 20 --request-overlap --request-frequency 10 --num-requests 1000
--tmax-pair 10 --enable-pdb
\end{verbatim}

\section{Data Collection}
Simulations will record data into a SQLite database under a timestamped directory within the specified results-path.  For the EGP simulations one can interact with the database using the following code:

\begin{verbatim}
import sqlite3
conn = sqlite3.connect(<path_to_sim_data.db>)
c = conn.cursor()

# Showing all tables constructed
c.execute("SELECT name FROM sqlite_master WHERE type='table'")
for table_name in c.fetchall():
    print(table_name)

# Sample output:
('EGP_Creates_0',)
('EGP_OKs_0',)
('Node_EGP_Attempts_0',)
('Midpoint_EGP_Attempts_0',)

# Showing column number, name, and data type
c.execute("PRAGMA table_info(EGP_Creates_0)")
for column_data in c.fetchall():
    number, name, dtype, _, _, _ = column_data
    print(number, name, dtype)

# Sample output
0 Timestamp real
1 Node ID integer
2 Create ID integer
3 Create_Time real
4 Max Time real
5 Min Fidelity real
6 Num Pairs integer
7 Other ID integer
8 Priority integer
9 Purpose ID integer
10 Success integer

# Extracting Create's from table EGP_Creates_0
c.execute("SELECT * FROM EGP_Creates_0")
for p in c.fetchall():
    timestamp, nodeID, create_id, create_time, max_time, min_fidelity,
        num_pairs, otherID, priority, purpose_id, succ = p

# Extracting ok's
c.execute("SELECT *  FROM EGP_OKs_0")
for p in c.fetchall():
    timestamp, createID, originID, otherID, MHPSeq, logical_id, goodness,
        t_goodness, t_create, succ = p

# Extracting error's
c.execute("SELECT * FROM EGP_Errors_0")
for p in c.fetchall():
    timestamp, nodeID, error_code, succ = p

# Extracting node attempts
c.execute("SELECT * FROM Node_EGP_Attempts_0")
for p in c.fetchall():
    timestamp, nodeID, succ = p

# Extracting midpoint attempts
c.execute("SELECT * FROM Midpoint_EGP_Attempts_0")
for p in c.fetchall():
    timestamp, outcome, succ = p

# Reconstructing matrices:
c.execute("SELECT * FROM EGP_Qubit_States_0")
for p in c.fetchall():
    timestamp = p[0]
    nodeID = p[1]
    m_data = p[2:34]
    # Reconstruct matrix, note the '1j * ' used to include the imaginary component
    m = matrix([[m_data[i] + 1j * m_data[i+1]
                 for i in range(k, k+8, 2)]
                 for k in range(0, len(m_data), 8)])
    succ = p[-1]

\end{verbatim}
\end{document}
